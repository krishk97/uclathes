%
% appendix.tex
% Copyright (C) 2021 by Krish Kabra, <krish@kabra.com>.
%

\chapter{Deployment Cost Projections for Telemedicine}
\label{chap:telemedicine_cost_projections}

In order to calculate the estimated average deployment cost for the cheapest existing method (finger pulse oximeters), we use the following methodology:
\begin{enumerate}
    \item We identify the estimated user base numbers for telemedicine in the US using the numbers from \cite{kats_us_2020} and extend these up to 2027 using the compound annual growth rate (CAGR) of 15.8\% as suggested in \cite{polaris_us_2020}.
    
    \item We make the conservative assumption that all members of a given family would be active users of telemedicine services. Therefore, an estimate of the number of families using telemedicine services is given by: 
    \begin{equation}
        No.~of~Families = \frac{Number~of~Telemedicine~Users}{Avg.~Family~Size~in~the~US}
    \end{equation}
    We use the average family size of 3.15 from the U.S. Census Bureau's Current Population Survey \cite{cps_us_2020}.
    
    \item Assuming that one pulse oximeter costs \$20 (as observed from a survey of available units in the market), and assuming conservatively that one pulse oximeter has to be deployed per family, the cost of deployment is given by: 
    \begin{equation}
        Cost~of~Deployment = No.~of~Families~\times~Cost~per~Pulse~Oximeter~Unit
    \end{equation}
\end{enumerate}





