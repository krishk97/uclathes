
%
% introduction.tex
% Copyright (C) 2021 by Krish Kabra, <krish@kabra.com>.
%

\chapter{Introduction}

Filing services are one of the most user-visible
  parts of the operating system,
  so it is not surprising that 
  many new services are proposed
  by researchers
  and that a variety of third parties are interested in providing
  these solutions.
Of the many innovations which have been proposed,
  very few have become widely available
  in a timely fashion.
We believe this delay results from two deficiencies
  in practices of current file-system development.
First,
  file systems are large and difficult to implement.
This problem is compounded because
  no good mechanism exists to allow
  new services to build on those which already exist.
Second,
  file systems today are built around a few fixed interfaces
  which fail to accommodate the change and evolution inherent in
  operating systems development.
Today's filing interfaces vary from
  system to system,
  and even between point releases of a 
  single operating system.
These differences greatly complicate and therefore discourage
  third-party development and adoption of filing extensions.

These problems raise barriers to the widespread
  development, deployment, and maintenance of new filing
  services.
The thesis of this dissertation is that
  a layered,
  \emph{stackable} structure
  with an \emph{extensible} interface
  provides a much better methodology
  for file-system development.
We propose construction of filing services from a number
  of potentially independently developed modules.
By stackable,
  we mean that these modules are bounded by
  identical, or \emph{symmetric}, interfaces above and below.
By extensible, we mean that these interfaces
  can be independently changed by multiple parties,
  without invalidating existing or future work.

To validate this thesis we developed a
  framework supporting stackable file-systems
  and used that framework to construct several
  different filing services.
This dissertation describes the design,
  implementation,
  and evaluation of this system. \cite{chari2020diverse}


